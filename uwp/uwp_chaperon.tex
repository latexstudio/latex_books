\documentclass{book}

\usepackage{ctex}
\usepackage{listings}
\usepackage{geometry}

\usepackage{xcolor}
\usepackage{inconsolata}

\lstset{language=[Sharp]C,
	showspaces=false,
	showtabs=false,
	breaklines=true,
	showstringspaces=false,
	breakatwhitespace=true,
	escapeinside={(*@}{@*)},
	basicstyle=\codefont,             % use the font defined as \codefont
	stringstyle=\color{blue!70!black},
	commentstyle=\color{green!70!black},
	columns=fullflexible
}

\lstdefinestyle{csharpstyle}{
	basicstyle = \ttfamily,
	keywordstyle = \color{blue}\textbf,
	commentstyle = \color{gray},
	stringstyle = \color{green!70!black},
	stringstyle = \color{red},
	columns = fullflexible,
	numbers = left,
	numberstyle = \scriptsize\sffamily\color{gray},
	xleftmargin = 0.16\textwidth,
	xrightmargin = 0.16\textwidth,
	showstringspaces = false,
	float,
}

\usepackage{hyperref}
\hypersetup{
	pdfstartview={FitH},
	pdftitle={Universal Windows Platform Chaperon},
	pdfauthor={Roger Young},
	pdfsubject={Universal Windows Platform},
	pdfcreator={Roger Young},
	pdfproducer={Roger Young},
	pdfkeywords={Universal Windows Platform, UWP},
}
\usepackage{glossaries}
\makeglossaries

\title{Universal Windows Platform Chaperon}
\author{Roger Young}
\date{最后更新于:\today}

\newcommand{\sharecontract}{共享合约(Share Contract)}

\begin{document}
	\maketitle
	\tableofcontents
	
	\chapter{准备工作}
	
	\chapter{页面导航}
	
	\chapter{启动与激活}
	\section{应用与应用之间的通讯}
	应用与应用之间的通讯包括:通过\sharecontract \hyperref[subsection:sharedata]{为其他应用提供数据}、\hyperref[subsection:receivedata]{接受其他应用的数据}、\hyperref[subsection:copyandpaste]{常规的复制和粘贴}、\hyperref[subsection:draganddrop]{拖放}以及调用其它应用。
	
	 \sharecontract 是一种在应用之间快速共享\textbf{文本}、\textbf{链接}、\textbf{照片和视频}等数据的一种简便方法。 例如,用户可能希望使用社交网络应用与其好友共享网页,或者将链接保存在笔记应用中以供日后参考。
	 
	 应用可以通过两种方式支持“共享”功能。 首先,应用可以是提供用户要共享的内容的源应用。 其次,应用可以是用户选择作为共享内容目标的目标应用。 一个应用也可以既是源应用,也是目标应用。 如果你希望你的应用作为源应用共享内容,则需要确定你的应用可以提供的数据格式。
	 除了“共享”合约外,应用还可以集成用于传输数据的传统技术,如拖放或复制和粘贴。 除了在 UWP 应用之间通信外,这些方法还支持在桌面应用程序之间共享。
	 
	 \subsubsection{设置事件处理程序}
	 添加 DataRequested 事件处理程序以在用户每次调用共享时调用。 这种情况既会在用户点击应用中的控件(例如按钮或应用栏命令)时发生,也会在特定情况下(例如,如果用户完成某一关并获得了较高分数)自动发生。
	 
	 当发生 DataRequested 事件时,你的应用会收到 DataRequest 对象。 该对象包含 DataPackage,可用来提供用户要共享的内容。 你必须提供标题和要共享的数据。 描述是可选的,但建议提供。
	 
	 \subsubsection{选择数据}
	 你可以共享各种类型的数据,包括:
	 \begin{itemize}
	 	\item 纯文本
	 	\item 统一资源标识符 (URI)
		\item HTML
		\item 格式化的文本
		\item 位图
		\item 纯文本
		\item 文件
		\item 自定义开发人员定义的数据
	 \end{itemize}
	 DataPackage 对象可以包含其中一种或多种格式(可任意组合)。 下面的示例演示了如何共享文本。
	 
	 \subsubsection{设置属性}
	 当你打包数据进行共享时,可以给出各种可提供与共享内容相关的其他信息的属性。 这些属性帮助目标应用改善用户体验。 例如,当用户通过多个应用共享内容时,可提供说明帮助。 共享图像或指向网页的链接时,添加一个缩略图可为用户提供直观的参考。 有关详细信息,请参阅 DataPackagePropertySet。
	 除了 Title,所有属性都是可选的。 Title 属性具有强制性,必须进行设置。
	 
	 \subsubsection{启动共享 UI}
	 用于共享的 UI 由系统提供。 若要启动它,请调用 ShowShareUI 方法。
	 
	 \subsubsection{处理错误}
	 在大多数情况下,共享内容是一个简单直接的过程。 然而,意外总是有可能发生。 例如,应用可能需要用户选择用于共享的内容,但用户未选择任何内容。 若要处理这些情况,请使用 FailWithDisplayText 方法,它将在出现问题时向用户显示一条消息。
	 \subsubsection{通过委托延迟共享}
	 有时,它可能不宜立即准备用户要共享的数据。 例如,如果你的应用支持以几种可能的不同格式发送一个较大的图像文件,在用户进行选择之前,创建所有这些图像的效率非常低。
	 若要解决此问题,DataPackage 可以包含委托 - 一种在接收应用请求数据时调用的函数。 我们建议你在用户想要共享的数据很消耗资源时使用委托。
	 

	\subsection{共享数据}\label{subsection:sharedata}
	
	\subsection{接收数据}\label{subsection:receivedata}
	
	\subsection{Copy and paste}\label{subsection:copyandpaste}
	
	\subsection{Drag and drop}\label{subsection:draganddrop}
	
	\chapter{布局排版}
	
	\chapter{控件}
	
	\chapter{基本程序设计思路}
	
	
	
\end{document}
