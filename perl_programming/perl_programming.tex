% !TeX spellcheck = en_US
% !TeX encoding = UTF-8
\documentclass[11pt]{book}
\usepackage[top=3cm, bottom=3cm, left=3cm, right=3cm,
			headsep=10pt,a4paper] {geometry}
			
\usepackage{amsfonts}
\usepackage{amsmath}
\usepackage{amsthm}

\usepackage[english]{babel}

\usepackage{ctex}
\usepackage{enumitem}
\usepackage{environ}

\usepackage[colorlinks, CJKbookmarks]{hyperref}

\usepackage{inconsolata}
\usepackage{listings}
\usepackage{makeidx}
\usepackage{mathrsfs}

\usepackage{pgfplots}
\usepgfplotslibrary{polar}

\usepackage{tikz}
\usepackage{ulem}
\usepackage{xcolor}
\usepackage{xparse}

\usepackage{makeidx} % 务必放到最后

\newcommand{\programmingkeywords}[1]{\textit{{\color{blue}#1}}}
\newcommand{\inlineglossary}[1]{\uwave{#1}\index{#1}}
\newcommand{\programminglanguage}[1]{\textbf{#1}}


\title{Perl编程语言学习笔记}
\author{Roger Young}
\setCJKmainfont{楷体}

\date{编译时间:\today}

\begin{document}	
	\maketitle
	\tableofcontents
	\newpage
	
	\chapter{语言简介}
		\section{语言进化简史}
			\programminglanguage{Perl}的全称可以认为是“\textit{Practical Extraction and Report Language}”,也可以认为是“\textit{Pathological Eclectic Rubbish Listeria}”。一般来说,以大写的“P”开头的“Perl”是指Perl编程语言,以小写字母“p”开头的“perl”是指用来解释和运行Perl程序的程序(或曰解释器,interpreter)。
		\section{基本概念}
			
		\section{基本语法}
			本部分的基本语法是\programminglanguage{Perl}语法体系中最最基础的部分,包括:
			\begin{itemize}
				\item \programminglanguage{Perl}中所有的语句需要以“\textbf{;}”结束;
			\end{itemize}
		\section{标准或推荐的程序框架}
			对于一般的\programminglanguage{Perl}程序(当然也包括其他编程语言写得脚本、程序),建议采用utf-8编码,对于个人来说,常用的脚本框架如下:
			
\lstset{language=Perl} 
\begin{lstlisting}
#!/usr/bin/perl

use strict;
use utf8;
use warnings;

# 其他代码
\end{lstlisting}

		\section{基本编辑过程}
		\section{基本运行方法}
		
	\chapter{关键字}
	
	\chapter{输入输出}
		\section{输出}
		\section{获取用户输入}
		
	\chapter{基本数据结构}
		\programminglanguage{Perl}语言中没有类似于\programminglanguage{C}、\programminglanguage{C++}中的\programmingkeywords{int}、\programmingkeywords{long}等表示对象或者变量类型的关键字。在\programminglanguage{Perl}中变量的\uline{最基本类型}根据其所参与的运算来决定。\uline{最基本类型}是指数、字符、字符串等。当个\uline{最基本类型}类型组成的变量或对象,在\programminglanguage{Perl}中称为标量(Scalar)。由多个标量,可以组成数组、字典等其他相对复杂类型。在\programminglanguage{Perl}可以通过变量标识符前的sigil,来确定变量的类型:
		
		\begin{tabular}{|c|l|l|}
			\hline
			Sigil & 变量类型 & 举例 \\
			\hline
			\$ & 标量、scalar & \lstinline!my $days =30;! \\
			\hline
			@ & 数组 & \lstinline!my @months = ("January", "February", "March")! \\
			\hline
			\% & 字典、Hash & \lstinline!my \%months = ('Jan' = 'January','Feb' = February,'Mar' = 'March');! \\
			\hline
		\end{tabular}
	
		\section{数字}
			\programminglanguage{Perl}依赖于低层的\programminglanguage{C}库来表示数字。在\programminglanguage{Perl}中,所有的数字在内部都以双精度浮点数(double-precision floating-point value)保存的。 这意味着,\programminglanguage{Perl}对数字的精度和位数可能有些要求。同时,也意味着,\programminglanguage{Perl}中没有整数(integer)或者浮点数(floating-number)的概念。
			
			\programminglanguage{Perl}数字举例:
\lstset{language=Perl} 
\begin{lstlisting}
#!/usr/bin/perl

use strict;
use utf8;
use warnings;
use diagnostics;

use v5.10;

my $chhogori_altitude  = 8611;
my $mariana_trench_altitude = -11034;

my $chhogori_altitude_oct = 0206;
my $chhogori_altitude_hex = 0x21A3;
my $chhogori_altitude_bin = 0b10000110100011;
my $chhogori_altitude_bin_sep = 0b10_000_110_100_011;

my $planck_length = 1.616229e35;
my $planck_time = 5.39116E-44;
my $height = 1.73;
my $light_speed = 299_792_458;
my $sackur_tetrode_constant = -1.1517084;
my $atomic_mass = 1.660539040e-27;

use v5.22;
my $hex_exponent = 0x1f.0p3;
\end{lstlisting}

			在\programminglanguage{Perl}中,数字可以直接通过字面值输入。所谓“字面值”是指,在源程序中直接存在的、不需要在程序运行过程中经过运算或者IO操作就能的得到的值。
			
			通过数字字面值可以直接以十进制、八进制、十六进制、二进制表示整数。(计算机科学中的)整数的字面值可以是以下几种形式:
			\begin{itemize}
				\item 以阿拉伯数字开头的、连续的一个或多个阿拉伯数字;
				\item 以“+”或者“-”开头的,连续的一个或多个阿拉伯数字;
				\item 以“+0b”或者“-0b”开头的,连续的一个或多个“0”、“1”;
				\item 以“+0”或者“-0”开头的,连续的一个或多个小于8的阿拉伯数字;
				\item 以“+0x”或者“-0x”开头的,连续的一个或多个的字符,这些字符包括所有的阿拉伯数字、a、b、c、d、e、f、A、B、C、D、E、F;
				\item 以字符“\_”分隔的上述类型;
			\end{itemize}
			
			浮点数的表示,跟我们常见的写法一致。浮点数可以以“.”开始、可以以表示整数的“+”或者“-”开始。浮点数也可以包含字母“E”或者“e”来表示科学计数法。对于5.22版以后的\programminglanguage{Perl},也可以使用形如“\$hex\_exponent”的形式表示浮点数。
			
			\subsection{数字的一般操作}
			\programminglanguage{Perl}支持数字常用的加、减、乘、除等算法,其对应的运算符如下:
			
			\begin{tabular}{|c|c|l|l|}
				\hline
				操作 & 符号 & 举例 & 结果\\
				\hline
				加 & + & 3 + 2 & 5 \\
				\hline
				减 & - & 3 - 2 & 1 \\
				\hline
				乘 & * & 3 * 2 & 6 \\
				\hline
				除 & / & 3 / 2 & 1.5 \\
				\hline
				模\footnotemark & \% & 3 \% 2 & 1 \\
				\hline
				幂 & ** & 3 ** 2 & 9 \\
				\hline
			\end{tabular}
			\footnotetext{对于有负数参与的求模运算,其结果可能在不同的机器上不同。这意味着“\lstinline!say -10 \% 3;!”在可能产生不同的结果!!}
		
		\section{字符串}
			作为一门以字符处理为特长的语言,\programminglanguage{Perl}的字符串功能相当强大。字符串字面值可以有两种表示形式,分别是:
			
			\begin{itemize}
				\item 以英文单引号,“'”括起来的字符串:在这种情况下,除了“'”和“\textbackslash”外其他的字符都是代表其自身。(之所以是这两个,可能是因为,“'”已经用来分隔字符串了,如果想要在字符串中包含“'”,必须采用特殊的办法,而“\textbackslash”被选中了。相应的,字符串中可以通过“\textbackslash'”和“\textbackslash\textbackslash”来包含这两个特殊字符。)
				\item 以英文双引号,“"”括起来的字符串。这类字符串与其他编程语言中的字符串比较类似。字符串中含有特殊的命令,可以对字符串进行必要的操作。也是在后续描述中篇幅比较大的一种。
			\end{itemize}
		
		\subsection{转移字符}
			在\programminglanguage{Perl}中,以英文双引号,“"”括起来的字符串,英文称为“Double-Quoted String Literal”。Double-Quoted String Literal可以包含一些特殊意义的字符组合:
			\begin{tabular}{|c|l|}
				\hline
				转义字符组合 & 意义 \\
				\hline
				\textbackslash n & 换行 \\
				\hline
				\textbackslash r & 回车 \\
				\hline
				\textbackslash t & Tab\\
				\hline
				\textbackslash f & Formfeed \\
				\hline
				\textbackslash b & 退格 \\
				\hline
				\textbackslash a & alarm,响一声 \\
				\hline
				\textbackslash e & ASCII码中的Escape字符 \\
				\hline
				\textbackslash 0$ dd $ & 以八进制表示的ASCII字符 \\
				\hline
				\textbackslash x$ hh $ & 以十六进制表示的ASCII字符 \\
				\hline
				\textbackslash $ N \left\lbrace charactername \right\rbrace $ & charactername表示的Unicode字符 \\
				\hline
				\textbackslash c$ C $ & 控制字符 \\
				\hline
				\textbackslash\textbackslash & \textbackslash \\
				\hline
				\textbackslash " & " \\
				\hline
				\textbackslash l & 下一个字符小写 \\
				\hline
				\textbackslash L & 其后的字符小写,直到遇到\textbackslash E \\
				\hline
				\textbackslash u & 下一个字符大写 \\
				\hline
				\textbackslash U & 其后的字符小写,直到遇到\textbackslash E \\
				\hline
				\textbackslash Q & Quote nonword characters by adding a backslash until \textbackslash E \\
				\hline
				\textbackslash E & 终止\textbackslash L、\textbackslash U、\textbackslash Q \\
				\hline
			\end{tabular}
			\subsection{字符串中的变量插入:variable interpolated}
				可以直接在以英文双引号,“"”括起来的字符串中插入变量。
				
			\subsection{字符串程序示例}
\lstset{language=Perl} 
\begin{lstlisting}
#!/usr/bin/perl

use strict;
use utf8;
use warnings;
use diagnostics;

my $light_speed = 299_792_458;

print 'The speed of light in vacuum is important in physics.';
print "\n";
print "可能乱码呀,稍后解决:\N{SNOWMAN}、\x{2668}";
print "The speed of light in vacuum is important in physics.\a\n";
print "The speed of light in vacuum is $light_speed !"
\end{lstlisting}
		
		\section{基本数据类型之间的互换}
		\section{内置类型}
		
	\chapter{内置变量}
	
	\chapter{控制语句}
		\section{条件}
		\section{循环}
		
	\chapter{函数}
		\section{内置函数}
		
	\chapter{模块化机制和实践}
	
	\chapter{面向对象机制}
		\section{定义类}
		\section{封装}
		\section{继承}
		\section{多态}
		\section{接口}
	\chapter{标准库介绍}
	
	\chapter{常用应用场景}
		\section{网络}
		\section{字符操作}
		\section{数据库}
		\section{Json操作}
		\section{xml操作}
		\section{文件、目录操作}
		\section{进程管理}
		\section{线程管理}
		\section{正则表达式}
	

\end{document}