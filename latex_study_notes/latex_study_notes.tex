\documentclass[utf8]{book}

\usepackage{ctex}

% syntonly可以在导言区使用\syntaxonly命令,只排错,不生成DVI或PDF文档
\usepackage{syntonly}
% 如果需要生存文档,请将下一行注释掉
% \syntaxonly

\usepackage{verbatim}
\usepackage{fancyvrb}
\usepackage{listings}
\usepackage{graphicx}

\newcommand{\latexcommand}[1]{\textbf{\textbackslash #1}}
\newcommand{\latexpackage}[1]{\textbf{\underline{#1}}}


\title{\LaTeX 学习笔记}
\author{Roger Young\thanks{Email: rogeryoung@outlook.com}}
\date{\today}

\begin{document}
	\maketitle
	
	\frontmatter
	
	\tableofcontents
	
	\mainmatter
	
	\chapter{基础知识}
	\section{编码实践}
	中文编码将以使用ctex宏包,文档类使用UTF-8编码,并且使用xelatex命令编译。
	\section{\LaTeX 中的字符}
	\LaTeX 文档源代码中,\textbf{空格键}和\textbf{Tab键}输入的空白字符被视为“空格”。连续的多个空白字符被视为一个空格。每一行开头的空格忽略不计。
	
	行末的回车视为一个空格,所导致的效果一般是换行,但不分段。连续的两个回车,也就是空行,会将文字分段。多个连续的(或者以空白字符分隔的)空行会被认为是一个空行。也可以在行末使用命令\latexcommand{par} 强制分段。
	
	\LaTeX 原文档使用“\%”字符来表示注释。在这个字符到其后第一个回车符之前的字符都将被忽略。
	
	\subsection{\LaTeX 的特殊字符的输入}
	\begin{tabular}{|c|c|c|}
		\hline
		特殊字符 & \LaTeX 中特殊意义 & 输入方式 \\
		\hline
		\# &  & \\
		\hline
		\$ & 用于排版\textbf{行内}数学公式 & \\
		\hline
		\% & & \\
		\hline
		\& & 用于输入表格,分隔每列 & \\
		\hline
		\{ & & \\
		\hline
		\}& & \\
		\hline
		\_ & 数学公式中用于表示下标 &  \\
		\hline
		\^ & 数学公式中用于表示上标 & \latexcommand{\^} \\
		\hline
		\textbackslash  & 输入特殊字符,输入命令 & \latexcommand{textbackslash} \\
		\hline
		\textbackslash [  & 输入公式块 & \latexcommand{textbackslash} [ \\
		\hline
	\end{tabular}
	
	\subsection{特殊标点符号的输入}
		\LaTeX 中有三种长度的“横线”可以使用:连字号(-)、短破折号(--)和长破折号(---)。连字号主要用来组成复合词(father-in-law)、短破折号用来表示数字范围(2007--2011 )、长破折号作为破折号使用(Yes---or No?)。
	
		省略号(\ldots)的输入可以采用命令\latexcommand{ldots}、\latexcommand{dots},这两个命令等价。
	
		波浪号有两种:其中“\~ ”可以使用命令“\latexcommand{textbackslash}\~”输入,$ \sim $ 可以使用\latexcommand{sim}输入。
		
	\subsection{文本强调}\label{text_emphasis}
		强调文本的方式有很多中,比如可以采用\underline{下划线},可以采用\textit{斜体},或者\textbf{加粗}。\underline{对于比较长的部分采用默认的下划线容易引起一些问题,比如无法换行,blalalalalalala},不同的单词可能产生不同高低的下划线。这种情况可以借用\latexpackage{ulem}包来解决,可以采用\latexcommand{uline}轻松生成自动换行的下划线。
		
	\section{交叉引用和脚注}
		引用是\LaTeX 很强大的功能之一。在可以交叉引用的地方,可以使用\latexcommand{label}命令。然后在其他地方通过命令\latexcommand{ref}和\latexcommand{pageref}来引用。有关文本强调的备份,请参考第\pageref{text_emphasis}页\ref{text_emphasis},访问文本强调部分。而制作脚注,可以参考脚注\footnote{直接在文中插入\latexcommand{footnote}即可。}。而对于某些不能正确生成脚注的地方,比如表格环境,可以先在需要插入脚注的地方,使用命令\latexcommand{footnotemark}为脚注计数,然后再在合适的位置用命令\latexcommand{footnotetext}生成脚注。
		
		\begin{tabular}{l}
			\hline
			“天地玄黄,宇宙洪荒。日月盈昃,辰宿列张。”\footnotemark \\
			\hline
		\end{tabular}
		\footnotetext{表格里的名句出自《千字文》。}
		
	\section{特殊环境}
		\subsection{引用环境}
		\latexcommand{quote}用于引用较短的文字,首行不缩进。
		\begin{quote}
			冯唐易老,李广难封。屈贾谊于长沙,非无圣主;窜梁鸿于海曲,岂乏明时?所赖君子见机,达人知命。老当益壮,宁移白首之心?穷且益坚,不坠青云之志。酌贪泉而觉爽,处涸辙以犹欢。
		\end{quote}
		\latexcommand{verse}适用于诗歌排版,首行悬挂缩进。
		\begin{verse}
			噫吁嚱,危乎高哉!
			蜀道之难,难于上青天!
			蚕丛及鱼凫,开国何茫然!
			尔来四万八千岁,不与秦塞通人烟。
			西当太白有鸟道,可以横绝峨嵋巅。
			地崩山摧壮士死,然后天梯石栈方钩连。
			上有六龙回日之高标,下有冲波逆折之回川。
			黄鹤之飞尚不得过,猿猱欲度愁攀援。
			青泥何盘盘,百步九折萦岩峦。
			扪参历井仰胁息,以手抚膺坐长叹。
			问君西游何时还?畏途巉岩不可攀。
			但见悲鸟号古木,雄飞从雌绕林间。
			又闻子规啼夜月,愁空山。
			蜀道之难,难于上青天,使人听此凋朱颜。
			连峰去天不盈尺,枯松倒挂倚绝壁。
			飞湍瀑流争喧豗,砯崖转石万壑雷。
			其险也若此,嗟尔远道之人,胡为乎来哉。
			剑阁峥嵘而崔嵬,一夫当关,万夫莫开。
			所守或匪亲,化为狼与豺。
			朝避猛虎,夕避长蛇,
			磨牙吮血,杀人如麻。
			锦城虽云乐,不如早还家。
			蜀道之难,难于上青天,侧身西望长咨嗟。
		\end{verse}
		
		\latexcommand{quotation}适用于打断文字,并且进行首行缩进。
		逍遥游:
		\begin{quotation}
			北冥有鱼,其名为鲲。鲲之大,不知其几千里也;化而为鸟,其名为鹏。鹏之背,不知其几千里也;怒而飞,其翼若垂天之云。是鸟也,海运则将徙于南冥。南冥者,天池也。
			《齐谐》者,志怪者也。《谐》之言曰:“鹏之徙于南冥也,水击三千里,抟扶摇而上者九万里,去以六月息者也。”野马也搜索,尘埃也,生物之以息相吹也。天之苍苍,其正色邪?其远而无所至极邪?其视下也,亦若是则已矣。
			且夫水之积也不厚,则其负大舟也无力。覆杯水于坳堂之上,则芥为之舟;置杯焉则胶,水浅而舟大也。风之积也不厚,则其负大翼也无力。故九万里,则风斯在下矣,而后乃今培风;背负青天,而莫之夭阏者,而后乃今将图南。
			蜩与学鸠笑之曰:“我决起而飞,抢榆枋而止,时则不至,而控于地而已矣,奚以之九万里而南为?”适莽苍者,三餐而反,腹犹果然;适百里者,宿舂粮;适千里者,三月聚粮。之二虫又何知!
			小知不及大知,小年不及大年。奚以知其然也?朝菌不知晦朔,蟪蛄不知春秋,此小年也。楚之南有冥灵者,以五百岁为春,五百岁为秋;上古有大椿者,以八千岁为春,八千岁为秋。此大年也。而彭祖乃今以久特闻,众人匹之,不亦悲乎?
			汤之问棘也是已。穷发之北,有冥海者,天池也。有鱼焉,其广数千里,未有知其修者,其名为鲲。有鸟焉,其名为鹏,背若泰山,翼若垂天之云;抟扶摇羊角而上者九万里,绝云气,负青天,然后图南,且适南冥也。斥鷃笑之曰:“彼且奚适也?我腾跃而上,不过数仞而下,翱翔蓬蒿之间,此亦飞之至也。而彼且奚适也?”此小大之辩也。
			故夫知效一官、行比一乡、德合一君、而征一国者,其自视也,亦若此矣。而宋荣子犹然笑之。且举世誉之而不加劝,举世非之而不加沮,定乎内外之分,辩乎荣辱之境,斯已矣。彼其于世,未数数然也。虽然,犹有未树也。夫列子御风而行,泠然善也,旬有五日而后反。彼于致福者,未数数然也。此虽免乎行,犹有所待者也。若夫乘天地之正,而御六气之辩,以游无穷者,彼且恶乎待哉?故曰:至人无己,神人无功,圣人无名。
			尧让天下于许由,曰:“日月出矣,而爝火不息;其于光也,不亦难乎?时雨降矣,而犹浸灌;其于泽也,不亦劳乎?夫子立而天下治,而我犹尸之;吾自视缺然,请致天下。”许由曰:“子治天下,天下既已治也;而我犹代子,吾将为名乎?名者,实之宾也;吾将为宾乎?鹪鹩巢于深林,不过一枝;偃鼠饮河,不过满腹。归休乎君,予无所用天下为!庖人虽不治庖,尸祝不越樽俎而代之矣!”
			肩吾问于连叔曰:“吾闻言于接舆,大而无当,往而不反。吾惊怖其言。犹河汉而无极也;大有迳庭,不近人情焉。”连叔曰:“其言谓何哉?”曰:“藐姑射之山,有神人居焉。肌肤若冰雪,淖约若处子,不食五谷,吸风饮露,乘云气,御飞龙,而游乎四海之外;其神凝,使物不疵疠而年谷熟。吾以是狂而不信也。”连叔曰:“然。瞽者无以与乎文章之观,聋者无以与乎钟鼓之声。岂唯形骸有聋盲哉?夫知亦有之!是其言也犹时女也。之人也,之德也,将旁礴万物以为一,世蕲乎乱,孰弊弊焉以天下为事!之人也,物莫之伤:大浸稽天而不溺,大旱金石流,土山焦而不热。是其尘垢秕糠将犹陶铸尧舜者也,孰肯以物为事?”宋人资章甫而适诸越,越人断发文身,无所用之。尧治天下之民,平海内之政,往见四子藐姑射之山,汾水之阳,窅然丧其天下焉。
			惠子谓庄子曰:“魏王贻我大瓠之种,我树之成,而实五石。以盛水浆,其坚不能自举也。剖之以为瓠,则瓠落无所容。非不呺然大也,吾为其无用而掊之。”庄子曰:“夫子固拙于用大矣。宋人有善为不龟手之药者,世世以洴澼絖为事。客闻之,请买其方百金。聚族而谋曰:‘我世世为洴澼絖,不过数金,今一朝而鬻技百金,请与之。’客得之,以说吴王。越有难,吴王使之将,冬,与越人水战,大败越人。裂地而封之。能不龟手一也,或以封,或不免于洴澼絖,则所用之异也。今子有五石之瓠,何不虑以为大樽,而浮于江湖,而忧其瓠落无所容?则夫子犹有蓬之心也夫!”
			惠子谓庄子曰:“吾有大树,人谓之樗。其大本拥肿而不中绳墨,其小枝卷曲而不中规矩,立之涂,匠人不顾。今子之言大而无用,众所同去也。”庄子曰:“子独不见狸狌乎?卑身而伏,以候敖者;东西跳梁,不辟高下;中于机辟,死于罔罟。今夫斄牛,其大若垂天之云。此能为大矣,而不能执鼠。今子有大树,患其无用,何不树之于无何有之乡,广莫之野,彷徨乎无为其侧,逍遥乎寝卧其下。不夭斤斧,物无害者,无所可用,安所困苦哉!”
		\end{quotation}
	\subsection{代码环境}LaTeX
		\begin{verbatim}
#include <iostream>
int main()
{
	std::cout << "Hello, LaTeX" << std::endl;
	return 0;
}
		\end{verbatim}
	

	\backmatter
		
	\appendix
	
\end{document}