\documentclass[utf8]{book}

\usepackage{ctex}

% syntonly可以在导言区使用\syntaxonly命令,只排错,不生成DVI或PDF文档
\usepackage{syntonly}
% 如果需要生存文档,请将下一行注释掉
% \syntaxonly



\begin{document}
	
	\newcommand{\latexcommand}[1]{\textbf{\textbackslash #1}}
	
	\chapter{基础知识}
	\section{编码实践}
	中文编码将以使用ctex宏包,文档类使用UTF-8编码,并且使用xelatex命令编译。
	\section{\LaTeX 中的字符}
	\LaTeX 文档源代码中,\textbf{空格键}和\textbf{Tab键}输入的空白字符被视为“空格”。连续的多个空白字符被视为一个空格。每一行开头的空格忽略不计。
	
	行末的回车视为一个空格,所导致的效果一般是换行,但不分段。连续的两个回车,也就是空行,会将文字分段。多个连续的(或者以空白字符分隔的)空行会被认为是一个空行。也可以在行末使用命令\latexcommand{par} 强制分段。
	
	\LaTeX 原文档使用“\%”字符来表示注释。在这个字符到其后第一个回车符之前的字符都将被忽略。
	
	\subsection{\LaTeX 的特殊字符的输入}
	\begin{tabular}{|c|c|c|}
		\hline
		特殊字符 & \LaTeX 中特殊意义 & 输入方式 \\
		\hline
		\# &  & \\
		\hline
		\$ & 用于排版\textbf{行内}数学公式 & \\
		\hline
		\% & & \\
		\hline
		\& & 用于输入表格,分隔每列 & \\
		\hline
		\{ & & \\
		\hline
		\}& & \\
		\hline
		\_ & 数学公式中用于表示下标 &  \\
		\hline
		\^ & 数学公式中用于表示上标 & \latexcommand{\^} \\
		\hline
		\textbackslash  & 输入特殊字符,输入命令 & \latexcommand{textbackslash} \\
		\hline
		\textbackslash [  & 输入公式块 & \latexcommand{textbackslash} [ \\
		\hline
	\end{tabular}
	
	\subsection{特殊标点符号的输入}
		\LaTeX 中有三种长度的“横线”可以使用:连字号(-)、短破折号(--)和长破折号(---)。连字号主要用来组成复合词(father-in-law)、短破折号用来表示数字范围(2007--2011 )、长破折号作为破折号使用(Yes---or No?)。
	
		省略号(\ldots)的输入可以采用命令\latexcommand{ldots}、\latexcommand{dots},这两个命令等价。
	
		波浪号有两种:其中“\~ ”可以使用命令“\latexcommand{textbackslash}\~”输入,$ \sim $ 可以使用\latexcommand{sim}输入。
		

		
		
	
\end{document}