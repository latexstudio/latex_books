% !TeX spellcheck = en_US
% !TeX encoding = UTF-8
\documentclass[11pt]{book}
\usepackage{amsmath}
\usepackage{amsthm}
\usepackage{ctex}

\usepackage{pgfplots}
\usepackage[colorlinks, CJKbookmarks]{hyperref}

\newtheorem{exercise}{\hspace{2em}\textbf{例}}[section]
\newtheorem{definition}{\hspace{2em}\textbf{定义}}

\usepackage{makeidx} % 务必放到最后

\title{概率论与数理统计学习笔记}
\author{Roger Young}
\setCJKmainfont{楷体}

\date{编译时间:\today}

\begin{document}
	\maketitle
	\tableofcontents
	\newpage
	
	\chapter{\LaTeX 基本使用说明}
	\section{入门常识}
	
	\LaTeX 会默认丢弃命令后的空白字符。如果希望在命令后的到一个空白字符,可以在命令后加上\{\}和一个空格,或者加上一个特殊的空白距离命令。
	\LaTeX 源文件中可以使用\%来表示注释,\LaTeX 在处理文件时,会忽略\%后的改行剩余文本。如果需要使用较长的注释(注释块),可以使用verbatim包。
		\section{页面布局}
				\subsection{分页、分段、断行和断字}
		有必要时\LaTeX 会进行必要的分页、断行和断字。在特殊情况下,需要使用命令指示\LaTeX 进行分页、分段、断行和断字。这些命令包括:
		\begin{itemize}
			\item $ \backslash \backslash $ :连着的两个反斜线指示\LaTeX 另起一行,但不另起一段。
			\item $ \backslash newline $ 命令:等同于$ \backslash \backslash $;
			\item $ \backslash \backslash* $ 命令:强行断行后,还禁止分页;
			\item $ \backslash newpage $ 命令:另起一新页。
			\item $ \backslash hyphenation\{word list\} $ :可以用来断字;
			\item $ \backslash- $:提示断字位置。
			\item $ \backslash linebreak $:强制断行;
			\item $ \backslash nolinebreak $:强制不断行;
			\item $ \backslash linebreak[priority] $:建议断行,priority可选值为0、1、2、3、4;
			\item $ \backslash nolinebreak[priority] $:建议不断行,priority可选值为0、1、2、3、4;
			\item $ \backslash cleardoublepage $ :开始新的偶数页;
			\item $ \backslash clearpage $:开始新页,并且导致当前浮动的表格、图片等都输出。;
			\item $ \backslash newpage $:开始新页;
			\item $ \backslash enlargethispage\left\lbrace size\right\rbrace  $:扩大当前页;
			\item $ \backslash pagebreak $:强制分页;
			\item $ \backslash nopagebreak $:强制不分页。
			
		\end{itemize}
		\subsection{页眉设置}
		\subsection{页脚设置}
		\begin{tabular}{|l|l|}
			\hline
			$ \backslash footnote[number]\left\lbrace text\right\rbrace $ & 在当前页尾插入编号的脚标 \\
			\hline
		\end{tabular}
		\section{字体设置}

		\section{参考文献引用}
		\section{\LaTeX 环境简介}
		\section{数学公式}
		\section{作图功能}
		\LaTeX 可以通过内置的picture环境来创建简单的图形。对于创建更复杂的图形可以尝试以下包:
		\begin{itemize}
			\item beamer
			\item pgf, Portable Graphics Format
			\item tikz, TikZ
		\end{itemize}
	
		一个picture环境可以通过“$ \backslash begin\left\{picture\right\}(x, y)$ ”和“$ \backslash end\left\{picture\right\}$ ”,或者“$ \backslash begin\left\{picture\right\}(x, y)(x_0, y_0)$ ”和“$ \backslash end\left\{picture\right\}$”来创建。其中$ x $、$ y $、$ x_0 $、$ y_0 $和$ \backslash unitlength $相关。$ (x, y) $ 指示\LaTeX 为picture预留的大小。可选的参数$ (x_0, y_0)$是为picture预留的方形的的左下角的坐标。
		可以通过“$ \backslash setlength$ ”命令来设置:
		\begin{center}
			$ \backslash setlength\left\lbrace \backslash unitlength \right\rbrace \left\lbrace 1.2cm \right\rbrace$
		\end{center}
	
		绘图命名的格式一般如下:
		\begin{center}
			$ \backslash put(x, y)\left\lbrace object\right\rbrace  $
		\end{center}
	或者
	\begin{center}
		$ \backslash multiput(x, y)(\Delta x, \Delta y)\left\lbrace n\right\rbrace\left\lbrace object\right\rbrace  $
	\end{center}
	贝塞尔曲线是个例外,其格式为:
		\begin{center}
		$ \backslash qbezier(x_1, y_1)(x_2, y_2)(x_3, y_3) $
	\end{center}


常用的画图命令如下:\\
\begin{center}
	\begin{tabular}{|c|l|c|}
		\hline
		类型 & 命令 & 说明 \\
		\hline
		线段 & $ \backslash put(x, y)\left \lbrace \backslash line(x_1, y_1)\left\lbrace length\right\rbrace \right\rbrace $ & \\
		\hline
		箭头 & $ \backslash put(x, y)\left \lbrace \backslash vector(x_1, y_1)\left\lbrace length\right \rbrace \right \rbrace $ & \\		
		\hline
		圆 & $ \backslash put(x, y)\left \lbrace \backslash circle\left \lbrace diameter\right \rbrace \right \rbrace  $ & \\
		\hline
		文本 & $ \backslash put(x, y)\left \lbrace \$\ text\ \$ \right\rbrace  $ & \\ 
		\hline
		公式 & $ \backslash put(x, y)\left \lbrace \$\ formulas\ \$ \right\rbrace  $ & \\ 
		\hline
		椭圆形 & $ \backslash put(x, y)\left \lbrace \backslash oval(x, y)\right \rbrace $ & \\ 
		\hline
		椭圆形 & $ \backslash put(x, y)\left \lbrace \backslash oval(x, y)[position]\right \rbrace $ & \\ 
		\hline
		贝叶斯曲线 & $ \backslash qbezier(x_1, y_1)(x_2, y_2)(x_3, y_3) $ & \\ 
		\hline
	\end{tabular}
\end{center}
\begin{tikzpicture}
\begin{axis} [axis lines=center]
\addplot [domain=-3:3, thick, smooth] { x^3 - 5*x };
\end{axis}
\end{tikzpicture}

	\chapter{基本概念汇总}
	\section{基本概念}
	\begin{itemize}
		\item \textbf{确定性现象}:在一定条件下,必然发生的现象。
		\item \textbf{统计规律性}:在大量重复试验或观察中所呈现出的固有的规律性。
		\item \textbf{随机试验}:具有以下三个特点的试验称为随机试验。
		\begin{itemize}
			\item 可以再相同的条件下重复地进行;
			\item 每次试验的可能结果不止一个,并且能事先明确试验的所有可能结果;
			\item 进行一次试验前,不能确定哪一个结果会出现。
		\end{itemize} 
	
	随机试验一般可以使用大写斜体字母表示,比如$ E $ 。
		\item \textbf{样本空间}:我们将随机试验$ E $ 的所有可能结果组成的集合称为$ E $ 的样本空间,记为$ S $  。样本空间中的元素,即$ E $ 的每个结果,称为\textbf{样本点}。
		\item 随机事件:简称事件,是指试验$ E$ 的样本空间$ S$ 的子集。再每次试验中,当且仅当这一子集的恶一个样本点出现时,称这一事件发生。
		\item 频率:在相同条件下,进行了$ n $ 次试验,在这$ n $ 次试验中,事件$ A $ 发生的次数$ n_A $称为事件$ A $ 发生的频数。比值$ \frac{n_A}{n} $称为事件A发生的频率,并记成$ f_n (A) $ 。
		\item 概率:设$ E $ 为随机试验,$ S $ 是它的样本空间,对于$ E $ 的每一事件$ A $ 赋予一个实数,记为$ P(A) $ ,称为事件$ A $的概率。概率是表示事件在一次试验中发生的可能性的大小的数。
		\item 等可能概型:具有以下两个特点的随机试验$ E $ 称为等可能概型。
		\begin{itemize}
			\item 随机试验的样本空间只包括两个元素;
			\item 试验中每个基本事件发生的可能性相同。
		\end{itemize}
		等可能概型是概率论发展初期研究的主要对象,所以也称为古典概型。
		\item 条件概率:设$ A $ ,$ B $ 是两个事件,且$ P(A)>0 $ ,称
		\begin{center}
			$ P(B|A)=\frac{P(AB)}{P(A)} $
		\end{center}
		为事件$ A $ 发生的条件下事件$ B $ 发生的条件概率。
		\item 随机变量:设随机试验的样本空间为$ S={e}$ ,$ X=X(e)$ 是定义在样本空间$ S $上的实值单值函数,对于任意实数$ x$ ,集合$ {e|X(e)\le x}$有确定的概率,则称$ X=X(e)$ 为随机变量。
		\item 离散型随机变量:可能取到的值是有限多个或者可列无限多个的随机变量。
		\item 离散型随机变量的分布律:设离散型随机变量$ X $ 所有可能取的值为$ x_k(k=1, 2, \dots)$ ,$ X $ 取各个可能值的概率,即事件$ {X=x_k} $ 的概率,为
		\begin{center}
			$ P\left\{X=x_k\right\}=p_k, k=1, 2, \dots $
		\end{center}
		则称该式为离散型随机变量$ X $ 的分布律。分布律也可以用表格的形式表示:
		\begin{center}
			\begin{tabular}{|c|c|c|c|c|c|}
				\hline
				$ X $ & $ x_1 $ & $ x_2 $ & \dots & $ x_n $ & \dots \\
				\hline 
				$ p_k $ & $ p_1 $ & $ p_2 $ & \dots & $ p_n $ & \dots \\
				\hline
			\end{tabular}
		\end{center}
	\end{itemize}

	\section{常用数学公式}
		\begin{equation}
			C_a ^r =\binom{a}{r}=\frac{a!}{r!(a-r)!}=\frac{a(a-1)\dots (a-r+1)}{r!}
		\end{equation}
	\section{重要公式的证明}
	\chapter{随机变量及其分布}
	\section{离散型随机变量及其分布律}
	\begin{tabular}{|l|c|c|}
		\hline
		常用分布 & 分布律 & 分布函数 \\
		\hline
		(0-1) 分布 & $
		P\left\lbrace X=k\right\rbrace =p^k\left( 1-p\right) ^{1-k},\hspace{1em}k=0,1 \hspace{1em}\left( 0<p<1\right) 
		$ & \\
		\hline
	\end{tabular}

\begin{tabular}{l*{6}{c}r}
	Team              & P & W & D & L & F  & A & Pts \\
	\hline
	Manchester United & 6 & 4 & 0 & 2 & 10 & 5 & 12  \\
	Celtic            & 6 & 3 & 0 & 3 &  8 & 9 &  9  \\
	Benfica           & 6 & 2 & 1 & 3 &  7 & 8 &  7  \\
	FC Copenhagen     & 6 & 2 & 1 & 3 &  5 & 8 &  7  \\
\end{tabular}
 



	\section{连续型随机变量及其概率密度}
	\section{多维随机变量及其分布}
	\section{随机变量的数学特征}
	\chapter{大数定律及中心极限定律}
	\chapter{参数估计}
	统计推断的基本问题可以分为两类,一类是估计问题,另一类是假设检验问题。
	\section{点估计}
	\begin{definition}
		设总体$X$的分布函数的形式已知,但它的一个或多个参数未知,借助于总体$X$的一个样本来估计总体未知参数的值的问题,称为参数的点估计问题。
	\end{definition}
点估计问题的一般提法如下:设总体$X$的分布函数$F(x;\theta)$的形式已知。$\theta$为待估参数,$X_1$,$X_2$,$\dots$,$X_n$是$X$的一个样本。$x_1$,$x_2$,$\dots$,$x_n$为相应的一个样本值。点估计的问题就是要构建一个适当的统计量$\hat{\theta}(X_1,X_2,\dots,X_n)$,用它的观测值$\hat{\theta}(x_1,x_2,\dots,x_n)$作为未知参数$\theta$的近似值,我们称$\hat{\theta}(X_1,X_2,\dots,X_n)$为$\theta$的估计量,称$\hat{\theta}(x_1,x_2,\dots,x_n)$为$\theta$的估计值。

由于估计量是样本的函数,因此对于不同的样本值,$\theta$的估计值一般是不相同的。同时,对于同一参数,用不同的估计方法求出的估计量可能也不相同。可以通过无偏性、有效性和相合性来评价估计量。
\subsection{无偏性}
\begin{definition}
	假设$X_1$,$X_2$,$\dots$,$X_n$是总体$X$的一个样本,$\theta\in\Theta$是包含在通体$X$的分布中的待估参数()$\Theta$是$\theta$的取值范围)。若估计量$\hat{\theta}(X_1,X_2,\dots,X_n)$为$\theta$的数学期望$E(\hat{\theta})$存在,且对于任意$\theta\in\Theta$有,
	\begin{center}
		$E(\hat{\theta})=\theta$
	\end{center}
则称$\hat{\theta}$是$\theta$的\textbf{无偏估计量}。
\end{definition}

在科学技术中将$E(\hat{\theta})-\theta$称为以$\hat{\theta}$作为$\theta$的估计的系统误差。无偏估计的实质意义就是无系统误差。

设总体$X$的均值为$\mu $,方差为$\sigma^2>0$均未知,不论总体$X$服从什么分布,样本均值$\overline{x}$都是总体均值$\mu $的无偏估计,样本方差$ S^2 =\frac{1}{n-1}\sum\limits_{i=1}^{n}(X_i-\overline{X})^2$是总体方差的无偏估计,但估计量$ \frac{1}{n}\sum\limits_{i=1}^{n}(X_i-\overline{X})^2$却不是$\sigma^2$的无偏估计。
\subsection{有效性}
方差ssh是随机变量取值与其数学期望的偏离程度的度量,所以无偏估计以方差小者为好。进而引入估计量有效性的概念。
\begin{definition}
	设$\hat{\theta}_1(X_1,X_2,\dots,X_n)$与$\hat{\theta}_2(X_1,X_2,\dots,X_n)$都是$\theta$的无偏估计量,若对于任意$\theta\in\Theta$,有
	\begin{center}
		$D(\hat{\theta}_1) \le D(\hat{\theta}_2)$
	\end{center}
	则至少对于某一个$\theta\in\Theta$,上式中的不等号成立,则称$D(\hat{\theta}_1)$较$D(\hat{\theta}_2)$有效。
\end{definition}
\subsection{相合性}
\begin{definition}
	设$\hat{\theta}(X_1,X_2,\dots,X_n)$为参数$\theta$的估计量,若对于任意$\theta\in\Theta$,当$n\to\infty$时,$\hat{\theta}(X_1,X_2,\dots,X_n)$以概率收敛于$\theta$则称有
	$\hat{\theta}$为$\theta$的相合估计量。
\end{definition}
相合性意味着随着样本量的增大,一个估计量的值是否可以稳定于待估参数的真值。相合性时对一个估计量的基本要求,若估计量不具有相合性,那么无论将样本量取得多大,都不能将参数$\theta$估计的足够准确。
	
	构建估计量的方法通常有两种:\textbf{矩估计法}和\textbf{最大似然估计法}。
	
	\subsection{矩估计法}
	设$X$为概率密度为$f(x;\theta_1,\theta_2,\dots,\theta_n)$ 的连续型随机变量,或者$X$为分布律为$P\left\lbrace X=x\right\rbrace =p(x;\theta_1,\theta_2,\dots,\theta_n)$的离散型随机变量,其中的$\theta_1,\theta_2,\dots,\theta_n$为待估参数,$X_1,X_2,\dots,X_n$是来自$X$的样本。假设总体$X$的前$k$阶矩
	\begin{center}
		\begin{tabular}{r l}
			$ \mu_1 = E(X^l) = \int{-\infty}^{\infty} x^l f(x;\theta_1,\theta_2,\dots,\theta_n) dx$ & $X$为连续型\\
			$ \mu_1 = E(X^l) = \sum\limits_{x \in R_x} x^l p(x;\theta_1,\theta_2,\dots,\theta_n)$  & $X$为离散型\\
		\end{tabular}
	\end{center}
	($l=1,2,\dots,k$,\ $R_x$是$X$可能的取值范围)存在。一般来说,它们是$\theta_1,\theta_2,\dots,\theta_n$的函数,基于样本矩
	\begin{center}
		$ A_l= \frac{1}{n}\sum\limits_{i=1}^n X_i^l $
	\end{center}
依概率收敛于相应的总体矩阵$\mu_l(l=1,2,\dots,k)$,样本矩的连续函数依概率收敛于相应的总体矩的连续函数,我们就用样本矩作为相应的总体矩的估计量,以样本矩的连续函数作为相应的总体矩的连续函数的估计量,这种估计方法称为矩估计法。

矩估计法的具体做法如下:






	\chapter{假设检验}
	\section{假设检验问题的$ P $ 值法}
	\begin{exercise}
	设总体$ X \sim N(\mu ,\sigma ^{2})$ ,$ \mu $ 未知,$ \sigma ^{2}=100 $,现有样本$ x_1 $,$ x_2 $,\dots ,$ x_{52} $,算得$ \overline{x} = 62.75$ 。现在来检验假设:
	\begin{center}
		$ H_0:\mu \le \mu _0=60,\ H_1 :\mu >60 $。
	\end{center}
\end{exercise}

\begin{proof}
	采用$ Z $ 检验法,检验统计量为:
	\begin{equation*}
		Z=\frac{\overline{x} - \mu_0}{\frac{\sigma}{\sqrt{n}}}
	\end{equation*}
	以数据代入,得$Z$的观察值为:
	\begin{equation*}
		Z=\frac{62.75 - 60}{\frac{10}{\sqrt{52}}}=1.983
	\end{equation*}
	概率
	\begin{equation*}
		P\left\lbrace Z \ge z_0\right\rbrace =P\left\lbrace Z\ge1.983\right\rbrace =1-\Phi(1.983)=0.0238
	\end{equation*}
	称之为$Z$检验法的右边检验的$p$值。
\end{proof}
\begin{definition}
	假设检验问题的$ p $ 值(probability value)是由检验统计量的样本观测值得出的原假设可能被拒绝的最小显著性水平。
	在现在计算机统计软件中,一般都会给出检验问题的$ p $ 值。按照$ p $ 值的定义,对于任意指定的显著性水平$ \alpha $,就有:
	
	\begin{enumerate}
		\item 若$ p $ 值$ \le \alpha $,则在显著性水平$ \alpha $ 下拒绝 $ H_0 $ ;
		\item 若$ p $ 值$ > \alpha $,则在显著性水平$ \alpha $ 下接受 $ H_0 $ ;
	\end{enumerate}
这种利用$p$值来确定是否拒绝 $ H_0 $ 的方法,称为$p$值法。
\end{definition}


	\chapter{方差分析}
	\chapter{回归分析}
	
	
\end{document}